%++++++++++++++++++++++定理环境+++++++++++++++++++++++++++++%
\makeatletter %将文献引用作为上标出现,增加括号,
\def\@cite#1#2{\textsuperscript{[{#1\if@tempswa #2\fi}]}}
\makeatother
%使用ntheorem宏包时,下列命令定义定理环境,如有不同要求请自行修改
\baselineskip=20pt % 指定行距为20pt
\theoremstyle{plain}
\theoremseparator{.}
\theorembodyfont{\normalfont}
\newtheorem{thm}{\hskip8mm 定理}[chapter]
\newtheorem{lem}[thm]{\hskip8mm 引理}
\newtheorem{prop}[thm]{\hskip8mm 命题}
\newtheorem{cor}[thm]{\hskip8mm 推论}
\newtheorem{remark}[thm]{\hskip8mm 备注}
\newtheorem{defn}{\hskip8mm 定义}[chapter]
\newtheorem{conj}{\hskip8mm 猜想}
\newtheorem{example}{\hskip8mm 算例}[chapter]
\theoremstyle{nonumberplain}
% 证明末尾自动添加加空心框.
\newenvironment{pf}{\textbf{证明.}}{\hfill $\square$\par}
% 证明末尾自动添加加实心框.
% \newenvironment{pf}{\textbf{证明.}}{\hfill $\blacksquare$\par}
\theoremseparator{:}
\theoremheaderfont{\heiti}
\theorembodyfont{\normalfont}
