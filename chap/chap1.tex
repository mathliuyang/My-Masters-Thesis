\chapter[绪论]{绪论}
% 本章主要介绍非线性分数阶薛定谔波动方程(NFSWEs)方程的研究意义、研究背景和发展现状, 以及本文的实际研究内容.

\section{研究意义}

分数阶微积分理论是研究任意阶微分和积分的理论,它是整数阶微积分理论的拓展.该理论起源于17世纪末,经过三个世纪的不懈努力,包括Riemann-Liouville、Grüwald-Letnikov、Caputo和Riesz在内的多种分数阶微积分理论被形成,并在专著\cite{samkoFractionalIntegralsDerivatives1993}中得到详细介绍.
由于分数阶微积分的物理及几何解释存在挑战,因此该领域在很长一段时间内停留在纯数学理论层面.但近几十年来,随着多个学科领域的研究发现,分数阶微分方程的保记忆性可优美地描述复杂问题,其描述精度超过整数阶微分方程.目前,分数阶微分方程已成功地应用于物理学、化学、生物学、水文学、混沌理论、复杂粘弹性材料、系统控制、信号处理、经济学等领域的问题,详见\cite{liIntroductionFractionalCalculus2015,HandbookDifferentialEquations2008,brychkovIndefiniteIntegrals2008,zhangMassBalanceBased2005,carrerasAnomalousDiffusionExit2001,hilferFRACTIONALCALCULUSREGULAR2000,liangRobustnessFractionalorderBoundary2007,maginSolvingFractionalOrder2009,zaslavskySelfsimilarTransportIncomplete1993,sunRandomorderFractionalDifferential2011}.

非线性分数阶薛定谔波动方程(NFSWEs)可以被看作是经典整数阶薛定谔波动方程的推广,
后者在Klein-Gordon方程的非相对论极限 \cite{tsutsumiNonrelativisticApproximationNonlinear1984,machiharaNonrelativisticLimitEnergy2002},等离子体中Langmuir波包络的近似 \cite{colinSemidiscretizationTimeNonlinear1998},
以及用于光子弹的正弦-戈登方程的调制平面脉冲近似 \cite{baoComparisonsSineGordonPerturbed2010,xinModelingLightBullets2000}等物理应用中具有广泛应用,并且已经得到深入研究,例如见 \cite{zhangConservativeNumericalScheme2003,baoUniformErrorEstimates2012,chengSeveralConservativeCompact2018,brugnanoClassEnergyconserving哈密顿ian2018}.
NFSWEs中的空间导数采用了分数阶拉普拉斯算子 $(-\Delta)^{\frac{\alpha}{2}}, \alpha \in(0,2)$,而不再是经典薛定谔波动方程中的二阶 $(\alpha=2)$ 空间导数.
由于分数阶拉普拉斯算子的非局部性质,NFSWEs能很好地描述许多经典薛定谔波动方程不能描述的新现象.
然而,分数阶拉普拉斯算子的非局部性质也给NFSWEs的解析求解带来了挑战.
此外, 与许多其他基于物理场景的微分模型一样, NFSWEs也具有很多的守恒性质. 
在某些领域, 保留原始微分方程的某些不变性质的能力已成为评估数值模拟成功与否的标准 \cite{liFiniteDifferenceCalculus1995}.
因此,对NFSWEs解的精确、高效、保结构数值模拟就显得尤为重要.
\section{研究背景与发展现状}
非线性薛定谔方程是非线性科学中普适性很强的一个基本方程, 在很多物理分支 有着广泛的应用. 它也引起了学者们的广泛关注, 基于有限差分法\cite{liFastEnergyConserving2018}、 有限元法\cite{karakashianSpacetimeFiniteElement1998}、 间断有限元法\cite{zhangConservativeLocalDiscontinuous2017} 、 谱方法\cite{gongConservativeFourierPseudospectral2017}的各类守恒型数值方法被不断提出. 
就有限差分方法而言, Bao等\cite{baoUniformErrorEstimates2012} 建立了整数阶薛定谔波方程在有限差分方法下的一致误差估计, 这里的误差界限适用于一般非线性的薛定谔波方程一维、 二维和三维的情形. 
Wang和Zhang\cite{wangAnalysisNewConservative2006} 针对一类薛定谔波方程初边值问题, 给出了一些新的保守有限差分格式. 它们的优点是有一些离散的能量分别是守恒的. 利用Leray-Schauder不动点定理证明了有限差分 格式解的存在性. 
在能量范数下证明了$\mathcal{O}(h^2+\tau^2)$阶差分解的唯一性、 稳定性和收敛性. Zhang等\cite{zhangConservativeNumericalScheme2003}研究了一类非线性整数阶薛定谔波方程的初界值问题, 提出了一种显式而有效的有限差分格式. 
这是一个具有离散保守律的四层格式, 并证明了其收敛性和稳定性. Li等\cite{liCompactFiniteDifference2012}对非线性整数阶薛定谔波方程的周期初值问题, 构造了一个紧凑有限差分格式. 
这是一个具有离散守恒定律的三层格式. 用能量法证明了该模型在最大范数下$\mathcal{O}(h^4+\tau^2)$ 阶的无条件稳定性和收敛性. Wang 等\cite{wangDiscretetimeOrthogonalSpline2011} 给出了离散时间正交样条的配置格式, 采用正交样条配置法结合有限差分法构造了这些格式. 
从理论上分析了这些方法的保守性、收敛性和稳定性. Guo等\cite{guoEnergyConservingLocal2015}采用局部不连续Galerkin方法对空间进行离散, Crank-Nicholson格式对时间进行离散, 建立起能量守恒的全离散格式. 

随着分数阶微积分的发展, 研究者开始重视分数阶模型的研究.
例如,Wang 和 Xiao \cite{wangCrankNicolsonDifference2013} 首先提出了一种 Crank-Nicolson 差分格式,该格式为耦合非线性分数阶 薛定谔 方程保留了离散质量,
然后,他们进一步提出了一种线性隐式格式,该格式保留了修改后的离散质量和能量\cite{wangLinearlyImplicitConservative2014}. 
Ran 和 Zhang \cite{ranConservativeDifferenceScheme2016} 提出了一种隐式差分格式和线性差分格式,分别为强耦合非线性分数阶 薛定谔 方程保留原始和修正的质量和能量. 
Wang 和 Huang \cite{wangEnergyConservativeDifference2015,wangConservativeLinearizedDifference2015} 推导出单三次分数阶 薛定谔 方程的能量和质量守恒 Crank-Nicolson 差分格式和线性差分格式.
Wang 等 \cite{wangSplitstepSpectralGalerkin2019} 提出了一种用于二维非线性空间分数阶 薛定谔 方程的分步谱 Galerkin 方法,该方法仅保留离散质量.

由于NFSWEs的非局部性和非线性,使其与经典的非线性薛定谔方程相比,理论和数值方法的研究相对较少,可参考的文献也比较有限.
针对模型NFSWEs,Ran 和 Zhang \cite{ranLinearlyImplicitConservative2016} 首先开发了一种三层线性隐式差分格式,它很好地保留了修改后的离散质量和能量. 
Li 和 Zhao \cite{liFastEnergyConserving2018} 考虑将 Crank-Nicolson 方法与 Galerkin 有限元方法相结合的守恒策略,并设计了具有合适循环预调节器的快速 Krylov 子空间求解器以节省计算成本. 
Cheng 和 Qin \cite{chengConvergenceEnergyconservingScheme2022} 开发了一种基于 SAV 方法的线性隐式守恒数值方案,该方案仅保留修改后的能量,而不保留质量.
Hu等 \cite{huEfficientEnergyPreserving2022} 分别在时间上应用 Crank-Nicolson、SAV 和 ESAV 方法,提出了三种能量守恒的谱 Galerkin 方法.
Zhang和Ran \cite{zhangHighorderStructurepreservingDifference2023} 提出并分析了基于三角SAV(T-SAV)方法的更高阶能量守恒差分格式.


尽管对于NFSWEs数值算法的研究已有不少工作, 但大多数方法仅关注一维情况, 在时间方向的精度没有高于二阶,并且/或者是完全隐式的.
此外,目前提出这些算法只能保修正后的能量和(或)质量守恒.
这意味着仍然存在许多开放问题,对于未来的研究,特别是在高维情况下,需要高效、准确、显式的数值方法,以及能够同时保持原始能量和质量的数值方法.

\section{研究内容}
基于以上研究的空白, 本篇论文主要从以下两个方面给出了数值求解NFSWEs的研究成果.

一方面,注意到在具有高阶精度的各种数值方法中,显式Runge-Kutta(RK)方法是很好的选择,因为它们属于单步方法,具有高阶精度和易于实现的特点.
然而,标准RK方法并不一定保持系统的期望物理特性.
为了解决这个问题,Ketcheson \cite{ketchesonRelaxationRungeKutta2019} 提出了松弛Runge-Kutta(RRK)方法,该方法保证相对于任何内积范数的守恒或稳定性.
随后,RRK技术在\cite{ranochaRelaxationRungeKutta2020}中扩展到一般的凸量上.因此,通过在每一步中添加一个乘以Runge-Kutta更新的松弛参数,可以强制执行相对于任何凸泛函的守恒、耗散或其他属性.
这些优势的代价是必须求解一个非线性代数系统来确定松弛参数.然而,对于非二次不变量的情况,作者们并未考虑构建显式的守恒方案.
幸运的是,不变能量二次化(IEQ)方法 \cite{yangLinearUnconditionallyEnergy2017, yangEfficientLinearSchemes2017} 和SAV方法 \cite{chengConvergenceEnergyconservingScheme2022} 可以通过变量变换将非二次能量转化为新变量的二次形式,而由此产生的新等效系统仍然保留了关于新变量的类似能量定律.
IEQ方法和SAV方法已被广泛应用于梯度流模型.例如,具有熔体对流\cite{chenEfficientNumericalScheme2019}的树状凝固相场模型,具有一般非线性势的梯度流动方程\cite{yangConvergenceAnalysisInvariant2020},外延薄膜生长模型\cite{chengHighlyEfficientAccurate2019},使用SAV \cite{gongArbitrarilyHighorderUnconditionally2019}的任意高阶无条件能量稳定格式.
有关IEQ和SAV的更多细节、扩展和改进,我们建议感兴趣的读者参考\cite{zhaoNumericalApproximationsPhase2017,shenScalarAuxiliaryVariable2018,liuExponentialScalarAuxiliary2020,chengMultipleScalarAuxiliary2018}.
受到这些发展的启发,本文通过结合SAV方法和显式RRK方法,为一维和二维NFSWEs开发了一种任意高阶的显式保结构数值格式.

另一方面,
注意到平均向量场(AVF)方法可以保持哈密顿系统的能量 \cite{buddGeometricIntegrationUsing1999,quispelNewClassEnergypreserving2008}.此外,最近提出的分区平均向量场(PAVF)方法不仅可以保持传统能量,还可以保持更多的守恒性质,并且已被用于构造哈密顿常微分方程的守恒数值方案,参见 \cite{caiPartitionedAveragedVector2018}.
这些研究基础使我们有可能实现我们的目标,而对 NFSWEs 的哈密顿结构的推导是构建保结构方法的成功关键.
据我们所知,目前很少有关注分数微分方程的哈密顿结构的研究.最近,王和黄 \cite{wangStructurepreservingNumericalMethods2018} 提出了带有分数拉普拉斯的泛函的变分导数,并将一维分数非线性薛定谔方程重新表述为一个哈密顿系统.傅和蔡 \cite{fuStructurepreservingAlgorithmsTwodimensional2020} 推导了二维分数Klein-Gordon-Schr{\"o}dinger方程的哈密顿形式,并随后发展了守恒数值方案.
基于此,我们首先推导了具有周期边界条件的二维 NFSWEs 的哈密顿形式,然后通过分区平均向量场加法(PAVF-P)方法,成功构建了能够同时守恒原始能量和质量的数值格式.

谱方法是非局部方法,这与分数阶薛定谔方程中的分数阶拉普拉斯算子的非局部性质相契合.傅里叶基函数是周期边界条件下的拉普拉斯算子的特征函数,且快速傅里叶变换 (FFT) 的使用使编程更加简便,可以大大减少计算量,提高计算效率.
因此,本文空间离散均选用傅里叶拟谱方法.

\section{论文安排}
本文结构安排如下:

第2章主要有3节. 首先介绍了分数阶微积分以及Fourier 谱方法, 
然后引进一些本文后续部分经常涉及的记号和引理.

第3章主要分6节. 在第 \ref{Section_SAVRRK: 2} 节中,我们通过引入一个标量辅助变量,将NFSWEs \eqref{eq_SAVRRK:1} 重新表述为一个等效系统.第 \ref{Section_SAVRRK: 3} 节通过对结果的重新表述应用傅立叶拟谱逼近,得到一个半离散的守恒系统.在第 \ref{Section_SAVRRK: 4} 节,我们通过在时间上应用松弛Runge-Kutta方法,提出了对重新表述的半离散系统的显式全离散方案,而所提出的标量辅助变量松弛Runge-Kutta(SAV-RRK)方案与标准标量辅助变量Runge-Kutta (SAV-RK) 方法具有相同的收敛阶.在第 \ref{Section_SAVRRK: 5} 节中,我们进一步估计引入的松弛系数,然后得到松弛方法的精度.第 \ref{Section_SAVRRK: 6} 节呈现了数值例子,以展示所提出方案的精度和守恒特性.在第 \ref{Section_SAVRRK: 7} 节中,我们进行简要总结.


第4章主要分4节. 在第 \ref{Section_PAVF: 2_1} 节,我们首先研究了带有周期边界条件的二维分数非线性薛定谔波方程 \eqref{eq_SAVRRK:1}-\eqref{eq_SAVRRK:2} 的哈密顿结构, 接着在 \ref{Section_PAVF: 2_2} 节中通过使用傅立叶拟谱方法在空间中逼近得到的哈密顿系统,推导出一个半离散的保守系统.
在第 \ref{Section_PAVF: 3} 节,通过使用PAVF-P方法对前述空间半离散系统进行离散化,我们得到了一类完全离散的保守数值方案,并证明了离散守恒定律.在第 \ref{Section_PAVF: 4} 节,我们提供了一维和二维数值示例,以演示理论结果.第 \ref{Section_PAVF: 5} 节总结了一些结论.

在第5章中.我们总结研究成果,列出一些进一步的研究成果并对今后的研究做一些展望.