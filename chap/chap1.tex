\chapter[绪论]{绪论}
% 本章主要介绍非线性分数阶薛定谔波动方程(NFSWEs)方程的研究意义、研究背景和发展现状, 以及本文的实际研究内容.

\section{研究意义}

分数阶微积分理论是研究任意阶微分和积分的理论,它是整数阶微积分理论的拓展.该理论起源于17世纪末,经过三个世纪的不懈努力,包括Riemann-Liouville、Grüwald-Letnikov、Caputo和Riesz在内的多种分数阶微积分理论被形成\cite{samkoFractionalIntegralsDerivatives1993}.
由于分数阶微积分的物理及几何解释存在挑战,因此该领域在很长一段时间内停留在纯数学理论层面.但近几十年来,随着多个学科领域的研究发现,分数阶微分方程的保记忆性可有效地描述某些复杂问题,其描述精度超过整数阶微分方程.
目前,分数阶微分方程已成功地应用于物理学、化学、生物学、水文学、混沌理论、复杂粘弹性材料、系统控制、信号处理、经济学等领域的问题\cite{liIntroductionFractionalCalculus2015,HandbookDifferentialEquations2008,brychkovIndefiniteIntegrals2008,zhangMassBalanceBased2005,carrerasAnomalousDiffusionExit2001,hilferFRACTIONALCALCULUSREGULAR2000,liangRobustnessFractionalorderBoundary2007,maginSolvingFractionalOrder2009,zaslavskySelfsimilarTransportIncomplete1993,sunRandomorderFractionalDifferential2011}.

非线性分数阶薛定谔波动方程(NFSWEs)可看作是经典整数阶薛定谔波动方程的推广,
后者在Klein-Gordon方程非相对论极限 \cite{tsutsumiNonrelativisticApproximationNonlinear1984,machiharaNonrelativisticLimitEnergy2002},等离子体中Langmuir波包络近似 \cite{colinSemidiscretizationTimeNonlinear1998},
以及光弹模型的Sine-Gordon方程的调制平面脉冲近似 \cite{baoComparisonsSineGordonPerturbed2010,xinModelingLightBullets2000}等物理场景中具有广泛应用,并且已得到深入研究\cite{zhangConservativeNumericalScheme2003,baoUniformErrorEstimates2012,chengSeveralConservativeCompact2018,brugnanoClassEnergyconservingHamiltonian2018}.
由于NFSWEs中的分数阶拉普拉斯算子 $(-\Delta)^{\frac{\alpha}{2}}, \alpha \in(0,2)$的非局部性, 使其能更好地描述许多经典薛定谔波动方程不能描述的新现象.
然而,这也给NFSWEs的数值求解带来了挑战.此外, 与许多其他基于物理场景的微分模型一样, NFSWEs也具有一些守恒性质. 
在某些领域, 保留原始微分方程的某些不变性质的能力已成为评估数值模拟成功与否的标准 \cite{liFiniteDifferenceCalculus1995}.
因此,对 NFSWEs 寻找精确、高效、保结构的数值模拟方法成为当前研究的重要课题之一.

\section{研究背景与发展现状}
非线性薛定谔方程是非线性科学中极具普适性的基本方程,在众多物理学领域都有着广泛的应用.其重要性引起了学术界的深切关注. 基于有限差分法\cite{liFastEnergyConserving2018}、 有限元法\cite{karakashianSpacetimeFiniteElement1998}、 间断有限元法\cite{zhangConservativeLocalDiscontinuous2017} 、 谱方法\cite{gongConservativeFourierPseudospectral2017}的各类保结构数值方法被不断提出. 
就有限差分方法而言, Bao等\cite{baoUniformErrorEstimates2012} 建立了整数阶薛定谔波动方程在有限差分方法下误差限适用于一维、 二维和三维情形的一致误差估计. 
Wang和Zhang\cite{wangAnalysisNewConservative2006} 针对一类薛定谔波动方程初边值问题, 给出了一些新的守恒格式. 利用Leray-Schauder不动点定理证明了有限差分格式解的存在性,
并在能量范数下证明了$\mathcal{O}(h^2+\tau^2)$阶差分解的唯一性、稳定性和收敛性. Zhang等\cite{zhangConservativeNumericalScheme2003}提出了一种四层显式的有限差分保结构格式,并证明了其收敛性和稳定性. 
Li等\cite{liCompactFiniteDifference2012}对具有周期性边界条件的非线性整数阶薛定谔波动方程, 构造了一个保结构的三层紧格式, 并用能量法证明了该格式在最大模范数下$\mathcal{O}(h^4+\tau^2)$ 阶的无条件稳定性和收敛性. 
Wang 等\cite{wangDiscretetimeOrthogonalSpline2011} 采用正交样条配置法结合有限差分法给出了离散时间正交样条的配置格式. 从理论上分析了这些格式的守恒性、收敛性和稳定性. 
Guo等\cite{guoEnergyConservingLocal2015}在时空方向上分别采用局部不连续Galerkin方法、Crank-Nicholson格式进行离散, 建立起能量守恒的全离散格式. 
随着分数阶微积分的发展, 研究者开始重视分数阶模型的研究.
例如,Wang 和 Xiao \cite{wangCrankNicolsonDifference2013} 首先提出了一种 Crank-Nicolson 差分格式,该格式为耦合非线性分数阶薛定谔方程保留了离散质量守恒,
然后,他们进一步提出了一种线性隐式格式,该格式保留了修改后的离散质量和能量守恒\cite{wangLinearlyImplicitConservative2014}. 
Ran 和 Zhang \cite{ranConservativeDifferenceScheme2016} 提出了一种隐式差分格式和线性差分格式,分别为强耦合非线性分数阶 薛定谔 方程保留原始和修正的质量和能量. 
Wang 和 Huang \cite{wangEnergyConservativeDifference2015,wangConservativeLinearizedDifference2015} 推导出立方分数阶薛定谔方程的能量和质量守恒 Crank-Nicolson 差分格式和线性差分格式.
Wang 等 \cite{wangSplitstepSpectralGalerkin2019} 提出了一种用于二维非线性空间分数阶 薛定谔 方程的分步谱 Galerkin 方法,该方法仅保留离散质量.
% 由于NFSWEs的非局部性和非线性,使其与经典的非线性薛定谔方程相比,理论和数值方法的研究相对较少,可参考的文献也比较有限.
针对模型NFSWEs,Ran 和 Zhang \cite{ranLinearlyImplicitConservative2016} 首先开发了一种三层线性隐式差分格式,它很好地保留了修改后的离散质量和能量. 
Li 和 Zhao \cite{liFastEnergyConserving2018} 考虑将 Crank-Nicolson 方法与 Galerkin 有限元方法相结合的守恒策略,并设计了具有合适循环预调节器的快速 Krylov 子空间求解器以节省计算成本. 
Cheng 和 Qin \cite{chengConvergenceEnergyconservingScheme2022} 开发了一种基于 SAV 方法的线性隐式守恒数值格式,该格式仅保留修改后的能量,而不保留质量.
Hu等 \cite{huEfficientEnergyPreserving2022} 分别在时间上应用 Crank-Nicolson、SAV 和 ESAV 方法,提出了三种能量守恒的谱 Galerkin 方法.
Zhang和Ran \cite{zhangHighorderStructurepreservingDifference2023} 提出并分析了基于三角SAV(T-SAV)方法的更高阶能量守恒差分格式.

如表\ref{tab:NFSWEs}所示, 尽管对于NFSWEs数值算法的研究已有不少工作, 但大多数方法仅关注一维情况, 在时间方向的精度没有高于二阶,并且/或者是完全隐式的.
此外,目前提出这些算法只能保修正后的能量和(或)质量守恒.
这意味着仍然存在许多开放问题,对于未来的研究,特别是在高维情况下,需要高效、准确、显式以及能够同时保持原始能量和质量的数值方法.
\begin{table}[htbp]
    \centering
    \small
    \caption{NFSWEs 国内外研究现状}
        \begin{tabular}{lccccc}
        \toprule
        \textcolor[rgb]{0,0,0}{\textbf{文献}} & \textcolor[rgb]{0,0,0}{\textbf{维度}} & \textcolor[rgb]{0,0,0}{\textbf{空间精度}} & \textcolor[rgb]{0,0,0}{\textbf{时间精度}} & \textcolor[rgb]{0,0,0}{\textbf{能量守恒}} & \textcolor[rgb]{0,0,0}{\textbf{质量守恒}} \\
        \midrule
        \textcolor[rgb]{0,0,0}{\textbf{\cite{ranLinearlyImplicitConservative2016}{2016,Ran}}} & 1 维   & 2 阶   & 2 阶   & 修正能量  & 修正质量 \\
        \midrule
        \textcolor[rgb]{0,0,0}{\textbf{\cite{liFastEnergyConserving2018}{2018,Li}}} & 1 维   & 2 阶   & 2 阶   & 修正能量  & 修正质量 \\
        \midrule
        \textcolor[rgb]{0,0,0}{\textbf{\cite{panFourthorderDifferenceScheme2022}{2022,Pan}}} & 1 维   & 4 阶   & 2 阶   & 无     & 无 \\
        \midrule
        \textcolor[rgb]{0,0,0}{\textbf{\cite{chengConvergenceEnergyconservingScheme2022}{2022,Cheng}}} & 2 维   & 2 阶   & 2 阶   & 修正能量  & 无 \\
        \midrule
        \textcolor[rgb]{0,0,0}{\textbf{\cite{huEfficientEnergyPreserving2022}{2022,Hu}}} & 1 维   & 谱精度   & 2 阶   & 修正能量  & 无 \\
        \midrule
        \textcolor[rgb]{0,0,0}{\textbf{\cite{zhangHighorderStructurepreservingDifference2023}{2023,Zhang}}} & 1 维   & 4 阶   & 2 阶   & 修正能量  & 无 \\
        \bottomrule
        \end{tabular}%
    \label{tab:NFSWEs}%
    \end{table}%
\section{研究内容}
基于以上研究的空白, 
% 本篇论文主要从以下两个方面展示了数值求解NFSWEs的研究成果.
一方面,注意到在诸多高精度数值方法中,显式龙格-库塔(RK)方法因其高阶精度和易于实现的特性而备受青睐.
然而,标准RK方法未必能够满足系统所期望保留的某些物理特性.为了解决这一问题,Ketcheson \cite{ketchesonRelaxationRungeKutta2019} 提出了松弛龙格库塔(RRK)方法,该方法能够保证系统在任何内积范数下的守恒或稳定性.
随后,RRK技术被扩展到一般的凸量上\cite{ranochaRelaxationRungeKutta2020}.
通过在RK更新的每一步中引入一个松弛参数,可以强制实现相对于任何凸泛函的守恒、耗散或其他属性.然而,这些优势的代价是必须求解一个非线性代数系统来确定松弛参数.不过,对于非二次不变量的情况,作者们并未考虑构建显式的守恒格式.
幸运的是,不变能量二次化(IEQ)方法 \cite{yangLinearUnconditionallyEnergy2017, yangEfficientLinearSchemes2017} 和SAV方法 \cite{chengConvergenceEnergyconservingScheme2022} 可以通过变量替换将非二次能量转化为新变量的二次形式,而由此产生的等效系统仍然保留了关于新变量的类似能量守恒性质.
IEQ方法和SAV方法已被广泛应用于梯度流模型\cite{zhaoNumericalApproximationsPhase2017,shenScalarAuxiliaryVariable2018,liuExponentialScalarAuxiliary2020,chengMultipleScalarAuxiliary2018}.例如,具有熔体对流\cite{chenEfficientNumericalScheme2019}的树状凝固相场模型,具有一般非线性势的梯度流动方程\cite{yangConvergenceAnalysisInvariant2020},外延薄膜生长模型\cite{chengHighlyEfficientAccurate2019},基于SAV \cite{gongArbitrarilyHighorderUnconditionally2019}的任意高阶无条件能量稳定格式.
% 有关IEQ和SAV的更多细节、扩展和改进,建议感兴趣的读者参考\cite{zhaoNumericalApproximationsPhase2017,shenScalarAuxiliaryVariable2018,liuExponentialScalarAuxiliary2020,chengMultipleScalarAuxiliary2018}.
受到这些发展的启发,本文结合SAV方法和显式RRK方法,为一维和二维NFSWEs构造了一种任意高阶的显式保结构数值格式.

另一方面,
注意到平均向量场(AVF)方法能够保持哈密顿系统的能量守恒 \cite{buddGeometricIntegrationUsing1999,quispelNewClassEnergypreserving2008}.
最近提出的分区平均向量场(PAVF)方法不仅可以保持传统能量守恒,还可以保持更多的守恒性质,并且已被用于构造哈密顿常微分方程的保结构格式\cite{caiPartitionedAveragedVector2018}.
而NFSWEs的哈密顿格式的推导是构造PAVF格式的前提.据所知,目前很少有关注分数阶微分方程哈密顿结构的研究. 最近,Wang和Huang \cite{wangStructurepreservingNumericalMethods2018} 提出了带有分数阶拉普拉斯的泛函的变分导数,
并将一维分数阶非线性薛定谔方程重构为一个哈密顿系统.Fu和Cai \cite{fuStructurepreservingAlgorithmsTwodimensional2020} 推导了二维分数阶Klein-Gordon-Schr{\"o}dinger方程的哈密顿形式,并给出了守恒格式.
基于这些思路,本文推导了具有周期边界条件的二维 NFSWEs 的哈密顿格式,然后通过分区平均向量场加(PAVF-P)方法,成功构建了能够同时守恒原始能量和原始质量的数值格式.

% 此外,谱方法是非局部方法,这与分数阶薛定谔方程中的分数阶拉普拉斯算子的非局部性质相契合.傅里叶基函数是周期边界条件下的拉普拉斯算子的特征函数,且快速傅里叶变换 (FFT) 的使用使编程更加简便,可以大大减少计算量,提高计算效率.
% 因此,本文空间离散均选用傅里叶拟谱方法.

\section{论文安排}
本文结构安排如下:

第2章主要有3节. 分别介绍了分数阶微积分、傅里叶谱方法以及分区平均向量场方法.

第3章主要分为6节.在第 \ref{Section_SAVRRK: 2} 节中,通过引入一个标量辅助变量,将NFSWEs \eqref{eq_SAVRRK:1} 重构为一个等效系统.
第 \ref{Section_SAVRRK: 3} 节至第 \ref{Section_SAVRRK: 4} 节通过对等效系统应用傅里叶拟谱方法以及显式RRK方法,构造了一个保结构的高阶显式格式.在
第 \ref{Section_SAVRRK: 5} 节中,进一步估计引入的松弛系数,以确定松弛方法的精度.第 \ref{Section_SAVRRK: 6} 节通过数值算例验证了所提出格式的精度和守恒特性,
并将其应用到其他类似方程以展示其普适性.最后,在第 \ref{Section_SAVRRK: 7} 节中进行了简要总结.

第4章主要分4节. 在第 \ref{Section_PAVF: 2_1} 节,首先推导了 NFSWEs \eqref{eq_SAVRRK:1}的哈密顿结构, 接着在 \ref{Section_PAVF: 2_2} 节中通过使用傅里叶拟谱方法在空间方向进行半离散.
在第 \ref{Section_PAVF: 3} 节,通过使用PAVF-P方法对前述空间半离散系统进行离散化,得到能够同时守恒原始能量和原始质量的数值格式,并证明了离散守恒定律.
在第 \ref{Section_PAVF: 4} 节,通过丰富的数值算例进一步验证了理论结果.第 \ref{Section_PAVF: 5} 节简要总结了一些结论.

第5章为总结与展望.