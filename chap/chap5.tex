
\chapter[总结与后续工作]{总结与后续工作}
\section{论文总结}

本文致力于解决非线性分数阶薛定谔方程(NFSWEs)的数值求解问题,旨在发展高效、准确、显式的数值方法,以及能够同时保持原始能量和质量守恒的算法。在当前研究中,尽管存在一些关于NFSWEs数值算法的研究,但主要集中在一维情况,且在时间方向的精度未能超过二阶,并且很多方法仅考虑了完全隐式的情况,留下了一系列开放问题。

研究的一方面首先通过SAV方法将NFSWEs的非二次能量转化为新变量的二次形式,然后结合显式Runge-Kutta(RK)方法和松弛技术,提出了一种任意高阶的显式保结构数值格式。
这一方法在每一步中引入松弛因子,以确保相对于任何内积范数的守恒性,
这一方法在一维和二维NFSWEs中都得到了应用,并在一系列应用中展现了高效的性能。

另一方面,成功推导了具有周期边界条件的二维NFSWEs的哈密顿形式,然后通过引入分区平均向量场(PAVF)方法,
并采用PAVF-P方法构建了能够同时守恒原始能量和质量的数值格式。

此外,本文使用了谱方法,即傅里叶拟谱方法,作为空间离散方法,充分利用了其非局部性质和傅里叶基函数的特性,通过快速傅里叶变换(FFT)提高了计算效率。

综合而言,本文在NFSWEs数值求解方面取得了新的研究成果,提出的方法在长时间仿真具有良好的适用性,为未来相关研究提供了有益的参考。
然而,仍然存在一些开放性问题,在今后的工作中,打算进一步研究以下几个方面:

尝试利用一些经典的或最新的PINNs模型来解决分数阶非线性薛定谔波方程的数值求解问题.为解决更高维的问题提供新的思路.

分析所提方法在长时间计算中的误差收敛性和稳定性,并与其他现有方法进行更详细的比较.这可能需要引入更多误差指标或评价标准,并考虑不同参数或初始条件下的影响.将尝试利用渐进分析(AA)或其他数学工具来估计所提方法在长时间计算中的误差上界,并利用谱收敛(SC)或其他数值工具来验证所提方法在空间方向上的收敛阶数.也将与其他一些典型方法进行系统比较,如显式 Runge-Kutta 方法、隐式 Runge-Kutta 方法、指数时间微分法等.

研究分数阶非线性 薛定谔 波方程的一些物理特性和应用背景,如孤子解、孤子相互作用、孤子稳定性等.这可能需要引入更多的分析方法,如逆散射变换(IST)、Bäcklund 变换(BT)、Painlevé 分析(PA)等,并考虑不同的边界条件和外部激励.

研究分数阶非线性 薛定谔 波方程的一些计算特性和优化,如并行计算、高效存储、快速求解等.这可能需要考虑更多的计算资源和工具,如GPU、MPI、CUDA等,并考虑不同的算法设计和实现方法.