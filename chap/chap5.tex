
\chapter[总结与展望]{总结与展望}
\section{本文总结}

本文致力于解决非线性分数阶薛定谔方程的数值求解问题,旨在发展高效、准确、显式的数值方法,以及能够同时保持原始能量和质量守恒的算法.在当前研究中,尽管存在一些关于NFSWEs数值算法的研究,但主要集中在一维情况,且在时间方向的精度未能超过二阶,并且很多方法仅考虑了完全隐式的情况,留下了一系列开放问题.

本文研究的一方面首先通过SAV方法将NFSWEs的非二次能量转化为新变量的二次形式,然后结合显式龙格库塔方法和松弛技术,提出了一种任意高阶的显式保结构数值格式.
并将这一方法应用到NFSWEs等类似方程的数值求解中.
另一方面,成功推导了具有周期边界条件的二维NFSWEs的哈密顿形式,并采用PAVF-P方法构建了能够同时守恒原始能量和质量的数值格式.

此外,本文使用了傅里叶拟谱方法作为空间离散方法,充分利用了其非局部性质和傅里叶基函数的特性,通过快速傅里叶变换(FFT)提高了计算效率.

综合而言,本文在NFSWEs数值求解方面取得了新的研究成果,提出的方法在长时间仿真具有良好的适用性,为未来相关研究提供了有益的参考.

\section{研究展望}
在本文的基础上及研究过程中,发现存在以下改进空间:

\begin{enumerate}[(1)]
    \item 推广到高维问题:目前研究主要集中在二维问题的差分格式上,未来可将这些方法推广至更高维度,以数值求解三维或更高维度的分数阶方程.
    
    \item 深入分析稳定性和收敛性:本文尚未对数值方法的收敛性及稳定性进行充分的分析.未来考虑对提出的数值方法进行更深入的稳定性和收敛性分析,以验证其可靠性和有效性.
    
    \item 提高PAVF-P方法的收敛阶:本文仅构建了PAVF-P的二阶的数值格式,未来的研究可以考虑构建更高阶的数值格式,以进一步提高数值方法的精度.
\end{enumerate}

% (4)探索新的数值方法和模型:未来的研究可以探索并应用新的数值方法和模型,如基于神经网络的PINNs方法,以解决非线性分数阶薛定谔方程的数值求解问题.
