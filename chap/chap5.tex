
\chapter[总结与后续工作]{总结与后续工作}
本文针对二维分数阶非线性 Schr{\"o}dinger 波方程,提出了两种数值方法,分别具有高阶显式守恒和结构保持的特点。我们通过理论分析和数值实验,验证了所提方法的有效性和优越性。我们的工作为分数阶微分方程的数值模拟提供了一些新的思路和技术。

在今后的工作中,我们打算进一步研究以下几个方面:

将所提方法推广到更高维或更复杂的分数阶非线性 Schr{\"o}dinger 波方程,如含有非局部或随机项的情形。这可能需要考虑更多因素如内存消耗、并行计算、误差控制等。我们将尝试利用快速傅里叶变换(FFT)或其他快速算法来加速空间离散化,并利用多重网格(MG)或其他预处理技术来加速时间离散化。我们也将探索一些自适应步长或自适应网格的策略,以提高数值精度和效率。

探索其他类型的守恒律或结构保持格式,如对称能量-动量张量守恒律或辛结构保持格式。这可能需要引入更多变量或约束条件,并考虑相容性、稳定性等问题。我们将尝试利用拉格朗日乘子法(LM)或其他变分方法来构造对称能量-动量张量守恒律,并利用哈密顿系统的辛结构来构造辛结构保持格式。我们也将分析这些格式在长时间计算中的误差增长和能量波动。

分析所提方法在长时间计算中的误差收敛性和稳定性,并与其他现有方法进行更详细的比较。这可能需要引入更多误差指标或评价标准,并考虑不同参数或初始条件下的影响。我们将尝试利用渐进分析(AA)或其他数学工具来估计所提方法在长时间计算中的误差上界,并利用谱收敛(SC)或其他数值工具来验证所提方法在空间方向上的收敛阶数。我们也将与其他一些典型方法进行系统比较,如显式 Runge-Kutta 方法、隐式 Runge-Kutta 方法、指数时间微分法等。

研究分数阶非线性 Schr{\"o}dinger 波方程的一些物理特性和应用背景,如孤子解、孤子相互作用、孤子稳定性等。这可能需要引入更多的分析方法,如逆散射变换(IST)、Bäcklund 变换(BT)、Painlevé 分析(PA)等,并考虑不同的边界条件和外部激励。

研究分数阶非线性 Schr{\"o}dinger 波方程与其他相关的分数阶微分方程之间的联系和区别,如分数阶非线性 Klein-Gordon 方程、分数阶非线性 Dirac 方程等。这可能需要引入更多的模型比较和模型选择方法,如信息准则(IC)、贝叶斯因子(BF)、似然比检验(LR)等,并考虑不同的数据拟合和参数估计方法。

研究分数阶非线性 Schr{\"o}dinger 波方程在其他领域的应用和推广,如量子力学、光学、生物学等。这可能需要考虑更多的实际问题和实验数据,并与其他领域的专家进行交流和合作。

研究分数阶非线性 Schr{\"o}dinger 波方程的一些数学特性和难点,如存在性、唯一性、正则性、爆破现象等。这可能需要引入更多的数学理论和技巧,如不动点定理(FPT)、最大值原理(MVP)、能量估计(EE)、浓度紧致性(CC)等,并考虑不同的假设和条件。

研究分数阶非线性 Schr{\"o}dinger 波方程的一些数值特性和挑战,如稳定性、收敛性、守恒律、结构保持等。这可能需要引入更多的数值分析和优化方法,如von Neumann 分析(vN)、CFL 条件(CFL)、Lax-Wendroff 定理(LW)、离散变分原理(DVP)等,并考虑不同的误差来源和控制方法。

研究分数阶非线性 Schr{\"o}dinger 波方程的一些计算特性和优化,如并行计算、高效存储、快速求解等。这可能需要考虑更多的计算资源和工具,如GPU、MPI、CUDA等,并考虑不同的算法设计和实现方法。