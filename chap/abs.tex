%++++++++++++++++++++++摘要部分+++++++++++++++++++++++++++++%
%+++++++++++++++++++++++++++++++++++++++++++++++++++++++++++%

\begin{ChineseAbstract}[副教授]%可选参数加入导师职称
\item {\heiti 摘要\ \ }
本文研究了二维分数阶非线性 薛定谔 波方程的数值方法和守恒性质.这类方程在非线性光学、传播动力学、水波动力学等物理问题中有广泛的应用.

首先,本文引入了一种新颖的显式守恒数值方法类别,该方法的阶数可达到任意高阶,用于解决一维和二维非线性分数薛定谔波方程。本研究所提出的方法基于标量辅助变量的思想。首先通过引入标量辅助变量,将研究对象的方程转化为一个等效的系统,随后将能量重新表述为三个二次项的和的形式。
在时间上,我们采用显式松弛Runge-Kutta方法,在空间上使用傅立叶拟谱离散化。经证明,由此产生的时空全离散方案在离散层面上能够保持重新表述的能量,精度达到机器精度。所提出的方法在长期计算中显著提升了数值稳定性,这一点通过数值实验证实。
值得注意的是,这一思想也可以轻松推广到其他类似的方程,如非线性分数波动方程和分数Klein-Gordon-Schr{\"o}dinger方程。

接下来,本文建立了一种结构保持的数值方法,用于二维分数阶非线性 薛定谔 波方程.我们的主要贡献是所提出的数值方法不仅完全保持原始能量,还完全保持原始质量.
我们首先利用分数阶 Laplacian 函数的变分原理,建立了问题的哈密顿结构.然后,我们将分区平均向量场加方法和傅里叶拟谱法应用于哈密顿系统,得到了一个全离散格式.所得全离散格式在离散层面上证明了完全保持原始质量和能量.为了比较,我们列出了其他一些数值方法.最后,我们给出一些数值实验来支持我们的理论结果.

\item {\heiti 关键词:\ \ } 分数阶非线性 薛定谔 波方程\qquad 哈密顿系统\qquad 标量辅助变量方法\qquad 松弛 Runge-Kutta 方法 \qquad 分区平均向量场方法 \qquad 傅里叶拟谱法
\end{ChineseAbstract}
\begin{EnglishAbstract}
\item {\bf Abstract\ \ } This paper investigates numerical methods and conservation properties for the two-dimensional fractional nonlinear 薛定谔 wave equation, which has widespread applications in physical problems such as nonlinear optics, propagation dynamics, and water wave dynamics.

First, we construct a high-order explicit conservative scheme using the scalar auxiliary variable (SAV) method and explicit relaxed Runge-Kutta method. We introduce a scalar variable to obtain an equivalent system and discretize in the spatial direction using the Fourier pseudospectral method, ensuring the second-order conservation law of the semi-discrete scheme. Then, we apply a fourth-order explicit relaxed Runge-Kutta method to the semi-discrete system, obtaining a scheme that accurately conserves discrete energy and improves the numerical stability of long-time computations.
Numerical experiments demonstrate the superiority of our scheme in long-time computations and validate the correctness of theoretical analysis.

Next, we develop a structure-preserving numerical method for the two-dimensional fractional nonlinear Schr{\"o}dinger wave equation. Our main contribution is that the proposed numerical method not only completely preserves the original energy but also completely preserves the original mass.
We first establish the Hamiltonian structure of the problem using the variational principle of the fractional Laplacian function. Then, we apply the partitioned averaged vector field method and Fourier pseudospectral method to the Hamiltonian system, obtaining a fully discrete scheme. The resulting fully discrete scheme proves the complete preservation of the original mass and energy at the discrete level.
To compare, we list some other numerical methods. Finally, we present some numerical experiments to support our theoretical results.
\item {\bf Keywords:} Fractional Nonlinear Schr{\"o}dinger Wave Equation\qquad Hamiltonian System\qquad Scalar Auxiliary Variable Method\qquad Relaxation Runge-Kutta Metho \qquad Partitioned Vector Field Method \qquad Fourier Spectral Method
\end{EnglishAbstract}