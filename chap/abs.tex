%++++++++++++++++++++++摘要部分+++++++++++++++++++++++++++++%
%+++++++++++++++++++++++++++++++++++++++++++++++++++++++++++%

\begin{ChineseAbstract}[副教授]%可选参数加入导师职称
\item {\heiti 摘要:\ \ }
本文研究了二维分数阶非线性薛定谔波方程的数值方法和守恒性质.这类方程在非线性光学、传播动力学、水波动力学等物理问题中有广泛的应用.

首先,针对具有周期性边界条件的二维分数阶非线性薛定谔波方程,通过引入标量辅助变量将其转化为一个等效的系统,其能量被重新表述为三个二次项的和的形式.然后在时空方向上分别采用显式松弛龙格库塔方法和傅里叶拟谱方法进行离散.
构造出在时间方向可达任意高阶的显式守恒数值格式.该思想很容易推广到分数Klein-Gordon-Schr{\"o}dinger方程等类似方程.数值实验验证了所提出的方法在长时间仿真中的数值稳定性.

为了能保额外的不变量守恒,基于分数阶拉普拉斯泛函的变分原理推导出了分数阶非线性薛定谔波方程的哈密顿结构.然后将分区平均向量场方法和傅里叶拟谱方法应用于哈密顿系统,构造了一个能同时保原始能量和质量守恒的全离散格式.
通过与其它方法进行数值比较,验证了本文所提出的方法具有更好的守恒性质.

\item {\heiti 关键词:\ \ } 分数阶非线性 薛定谔 波方程\qquad 哈密顿系统\qquad 标量辅助变量方法\qquad 松弛龙格库塔方法 \qquad 分区平均向量场方法 \qquad 傅里叶拟谱法
\end{ChineseAbstract}
\begin{EnglishAbstract}
\item {\bf Abstract\ \ } 
This paper investigates numerical methods and conservation properties of the two-dimensional fractional nonlinear Schr{\"o}dinger wave equation, which finds wide applications in physical problems such as nonlinear optics, propagation dynamics, and water wave dynamics.

Initially, for the two-dimensional fractional nonlinear Schr{\"o}dinger wave equation with periodic boundary conditions, it is transformed into an equivalent system by introducing a scalar auxiliary variable, where the energy is reformulated as a sum of three quadratic terms. Then, explicit conservative numerical schemes are developed in the temporal and spatial directions using explicit relaxed Runge-Kutta methods and Fourier spectral methods, respectively, achieving arbitrarily high-order accuracy in time. This approach can be readily extended to similar equations like the fractional Klein-Gordon-Schr{\"o}dinger equation. Numerical experiments confirm the numerical stability of the proposed method over long-time simulations.

To ensure additional invariant conservation, a Hamiltonian structure for the fractional nonlinear Schr{\"o}dinger wave equation is derived based on the variational principle with fractional Laplacian functional. Subsequently, employing the fractional vector calculus method and Fourier spectral methods on the Hamiltonian system, a fully discrete scheme is constructed that conserves both original energy and mass simultaneously. Numerical comparisons with other methods validate the superior conservation properties of the proposed approach.

\item {\bf Keywords:} Fractional Nonlinear Schr{\"o}dinger Wave Equation\qquad Hamiltonian System\qquad Scalar Auxiliary Variable Method\qquad Relaxation Runge-Kutta Metho \qquad Partitioned Vector Field Method \qquad Fourier Spectral Method
\end{EnglishAbstract}