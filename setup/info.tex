%+++++++++++++++++++++++题名页环境++++++++++++++++++++++++++++
%+++++++++++++++++++++++++++++++++++++++++++++++++++++++++++++
%题名页支持宏包
\usepackage{ulem}
\usepackage{ifthen}
%题名页页定义命令
%论文作者中文名
\newcommand{\thesicnuauthorc}{刘洋}
\newcommand{\SiCNUAuthorC}[1]{\renewcommand{\thesicnuauthorc}{#1}}
%论文作者英文名
\newcommand{\thesicnuauthore}{Liu Yang}
\newcommand{\SiCNUAuthorE}[1]{\renewcommand{\thesicnuauthore}{#1}}
%指导教师中文名
\newcommand{\thesicnusupervisorc}{冉茂华}
\newcommand{\SiCNUSupervisorC}[1]{
\newlength{\supervisorclength}
\settowidth{\supervisorclength}{#1}
\ifthenelse{\lengthtest{\supervisorclength = 0em}}
    {\renewcommand{\thesicnusupervisorc}{\hspace{3em}}}
    {\renewcommand{\thesicnusupervisorc}{#1}}
}
%指导教师英文名
\newcommand{\thesicnusupervisore}{Associate Prof. Ran Maohua}
\newcommand{\SiCNUSupervisorE}[1]{\renewcommand{\thesicnusupervisore}{#1}}
%专业名称中文
\newcommand{\thesicnumajorc}{数学}
\newcommand{\SiCNUMajorC}[1]{\renewcommand{\thesicnumajorc}{#1}}
%专业名称英文
\newcommand{\thesicnumajore}{Mathematics}
\newcommand{\SiCNUMajorE}[1]{\renewcommand{\thesicnumajore}{#1}}
%学号
\newcommand{\thesicnunumber}{20210801068}
\newcommand{\SiCNUNumber}[1]{
\newlength{\numberlength}
\settowidth{\numberlength}{#1}
\newlength{\numberlengthundefinde}
%单位代码
\settowidth{\numberlengthundefinde}{10636}
\ifthenelse{\lengthtest{\numberlength = 0em} \or \lengthtest{\numberlength < \numberlengthundefinde}}
{\renewcommand{\thesicnunumber}{\qquad\qquad}}
{\renewcommand{\thesicnunumber}{#1}}
}
%分类号
\newcommand{\thesicnuclass}{\zihao{5} O241.82}
\newcommand{\SiCNUClass}[1]{\renewcommand{\thesicnuclass}{#1}}
%密级
\newcommand{\thesicnusecret}{\zihao{5} 公开}
\newcommand{\SiCNUSecret}[1]{\renewcommand{\thesicnusecret}{#1}}
%研究方向
\newcommand{\thesicnuresearch}{偏微分方程数值解}
\newcommand{\SiCNUResearch}[1]{\renewcommand{\thesicnuresearch}{#1}}
% 所在学院
\newcommand{\thesicnucollege}{数学科学学院}
\newcommand{\SiCNUCollege}[1]{\renewcommand{\thesicnucollege}{#1}}
%论文中文题目
\newcommand{\thesicnutitlec}{非线性分数阶薛定谔波动方程的两类保结构数值方法}
%论文中文分行题目
\newcommand{\thesicnutitleca}{\bf 非线性分数阶薛定谔波动方程的}
\newcommand{\thesicnutitlecb}{\bf 两类保结构数值方法}
\newcommand{\SiCNUTitleC}[2][]{
\newlength{\titleca}
\settowidth{\titleca}{#2}
\newlength{\titlecb}
\settowidth{\titlecb}{#1}
\ifthenelse{\lengthtest{\titleca > 18em}}
    {\renewcommand{\thesicnutitleca}{标题过长, 请用可选参数}}%每行字数不超过~18
    {\renewcommand{\thesicnutitleca}{#2}}
\ifthenelse{\lengthtest{\titlecb > 18em}}
    {\renewcommand{\thesicnutitlecb}{标题过长, 请缩短标题}}
    {\renewcommand{\thesicnutitlecb}{#1}}
\ifthenelse{\lengthtest{\titleca > 18em} \and \lengthtest{\titlecb > 18em}}
    {\renewcommand{\thesicnutitleca}{标题过长, 请缩短标题}\renewcommand{\thesicnutitlecb}{标题过长, 请缩短标题}}
    {}
\renewcommand{\thesicnutitlec}{#2 #1}
\ifthenelse{\equal{#2}{}}
    {\renewcommand{\thesicnutitlec}{\hspace{17em}}}
    {}
}
%论文英文题目
\newcommand{\thesicnutitlee}{Two Classes of Structure-Preserving Numerical Methods for the Nonlinear Fractional Schr{\"o}dinger Wave Equations}
%论文英文分行题目
\newcommand{\thesicnutitleea}{\bf Two Classes of Structure-Preserving Numerical}
\newcommand{\thesicnutitleeb}{\bf Methods for the Nonlinear Fractional }
\newcommand{\thesicnutitleec}{\bf Schr{\"o}dinger Wave Equations}
\newcommand{\SiCNUTitleE}[3][]{
\newlength{\titleea}
\settowidth{\titleea}{#2}
\newlength{\titleeb}
\settowidth{\titleeb}{#1}
\newlength{\titleec}
\settowidth{\titleec}{#3}
\ifthenelse{\lengthtest{\titleea > 18em}}
    {\renewcommand{\thesicnutitleea}{标题过长, 请用可选参数}}%每行字数不超过~18
    {\renewcommand{\thesicnutitleea}{#1}}
\ifthenelse{\lengthtest{\titleeb > 18em}}
    {\renewcommand{\thesicnutitleeb}{标题过长, 请缩短标题}}
    {\renewcommand{\thesicnutitleeb}{#2}}
\ifthenelse{\lengthtest{\titleec > 18em}}
    {\renewcommand{\thesicnutitleec}{标题过长, 请缩短标题}}
    {\renewcommand{\thesicnutitleec}{#3}}
\ifthenelse{\lengthtest{\titleea > 18em} \and \lengthtest{\titleeb > 18em}\and \lengthtest{\titleec > 18em}}
    {\renewcommand{\thesicnutitleea}{标题过长, 请缩短标题}\renewcommand{\thesicnutitleeb}{标题过长, 请缩短标题}\renewcommand{\thesicnutitleec}{标题过长, 请缩短标题}}
    {}
\renewcommand{\thesicnutitlee}{#3#2 #1}
}

%论文提交日期
\newcommand{\thesicnudateofsubmit}{2024年\quad 月\quad 日\ \ \ \ \ \ \ \ }
\newcommand{\SiCNUDateOfSubmit}[1]{\renewcommand{\thesicnudateofsubmit}{#1}}
%论文答辩日期
\newcommand{\thesicnudateofdefence}{2024年\quad 月\quad 日\ \ \ \ \ \ \ \ }
\newcommand{\SiCNUDateOfDefence}[1]{\renewcommand{\thesicnudateofdefence}{#1}}
%\newcommand{}{}
%\newcommand{}[1]{\renewcommand{}{#1}}
