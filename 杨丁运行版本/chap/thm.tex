%++++++++++++++++++++++定理环境+++++++++++++++++++++++++++++%
%下面的命令\citep将参考文献以上标形式出现
\let\oldcitep\cite
\newcommand{\citen}[2][]{\textsuperscript{\oldcitep{#2}#1}}
\renewcommand{\cite}{\citen}
%下面的命令\citep将参考文献以上标形式出现
% \let\oldcitep\citep
% \newcommand{\citenp}[2][]{{\oldcitep{#2}#1}}
% \renewcommand{\citep}{\citenp}

%使用ntheorem宏包时,下列命令定义定理环境,如有不同要求请自行修改
\baselineskip=20pt % 指定行距为20pt
\theoremstyle{plain}
\theoremseparator{.}
\theorembodyfont{\normalfont}
\newtheorem{theorem}{\hskip8mm 定理}[chapter]
\newtheorem{lemma}[theorem]{\hskip8mm 引理}
\newtheorem{prop}[theorem]{\hskip8mm 命题}
\newtheorem{cor}[theorem]{\hskip8mm 推论}
\newtheorem{defn}{\hskip8mm 定义}[chapter]
\newtheorem{remark}{\hskip8mm 注}[chapter]
\newtheorem{conj}{\hskip8mm 猜想}
\newtheorem{example}{\hskip8mm 算例}[chapter]
\theoremstyle{nonumberplain}
% 证明末尾自动添加加空心框.
\newenvironment{proof}{ \textbf{证明. }}{\hfill $\square$\par}
% 证明末尾自动添加加实心框.
% \newenvironment{pf}{\hskip8mm \textbf{证明.}}{\hfill $\blacksquare$\par}
\theoremseparator{:}
\theoremheaderfont{\heiti}
\theorembodyfont{\normalfont}
